\chapter{Kurzfassung}

Regressionsmodelle sind ein wichtiges Werkzeug in der Datenanalyse. Sie finden Anwendung bei der statistischen Vorhersage von Werten. Um die erstellten Regressionsmodelle zu Überprüfen werden Visualisierungen benötigt. Die Visualisierungen können entweder aus einer Bibliothek angewandt oder eigenständig für den Anwendungsfall entwickelt werden. Dabei stößt man jedoch häufig auf Einschränkungen. Diese betreffen\linebreak schwere Integration in Projekte oder geringe Funktionalität und fehlende Interaktivität der Diagramme. Das Beheben dieser Probleme bedarf einer neuen Bibliothek zur Visualisierung von Regressionsmodellen. Das Ziel dieser Arbeit ist dabei die Entwicklung einer solchen Bibliothek, welche vier konkrete Diagramme beinhaltet. Diese sind Liniendiagramm, Streudiagramm, Blasendiagramm und Schnittdiagramm. Liniendiagramm und Streudiagramm finden häufig Verwendung und bieten einen einfachen Einstieg in die Entwicklung. Blasendiagramm und Schnittdiagramm bieten vielseitige Möglichkeiten der Interaktivität für die Anwender:innen. Ihre Implementierung bietet daher Einblicke zu den Schwierigkeiten die bei der Entwicklung komplexer Diagramme entstehen. Zur Entwicklung werden vier Bibliotheken verwendet: Matplotlib, Pandas und Scikit-learn. Die Arbeit beschreibt, wie diese Bibliotheken für die Entwicklung von Diagrammen eingesetzt werden. Als Ergebnis dieser Arbeit wird die fertige Bibliothek präsentiert.\linebreak Die verfügbaren Diagramme erlauben komplexe interaktive Analysen, welche mit bestehenden Bibliotheken nicht möglich sind. Eine eigene, zeitaufwändige Implementierung ist dementsprechend nicht notwendig. Zuletzt wird noch ein Ausblick auf mögliche\linebreak Erweiterungen der Bibliothek gegeben.